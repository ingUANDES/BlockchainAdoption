% cap3.tex

\chapter{Methodology} \label{capMethods} % la etiqueta para referencias


\section{Model}

Initial Idea: incorporate to \citeA{xandri_liquidity_2022} a second network that competes to the base network in terms of transaction-waiting cost. Networks should be weighted in the spirit of \citeA{jalan_incentive-aware_2024} and adoption of one network with respect to the other is inspired in \citeA{hinzen_bitcoins_2022}. We should also consider hackers following \citeA{crettez_general_2022} to differentiate networks in terms of waiting costs and hacking activity.

In terms of results, the approach by \cite{jalan_incentive-aware_2024} considers network stability in terms of \cite{bich_perfect_2023,bich_existence_2020}.

\section{A twist from Jalan et alii}

The objective function in \citeA{jalan_incentive-aware_2024} is given by
$$g_i(W,P)=w_i^T(\mu_i-Pe_i)-\gamma_iw_i^T\Sigma w_i$$

We would like to include a second network to transfer. Let us assume that the following decomposition is possible:

$$W=\tilde W+\hat W,$$

where $\tilde W$ represents the traditional network and $\hat W$ the blockchain.

{\bf Question:} What should be the new objective function?

Proposal: 

\begin{eqnarray}
    b_i(\tilde W,\hat W,\tilde P,\hat P)&=&\tilde w_i^T(\tilde \mu_i-\tilde Pe_i)-\gamma_i\tilde w_i^T\tilde \Sigma \tilde w_i\nonumber\\
            &+&\hat w_i^T(\hat \mu_i-\hat Pe_i)-\gamma_i\hat w_i^T\hat \Sigma \hat w_i\label{eq:b}
\end{eqnarray}

In this case the optimal choice is given by $\max_{\tilde W,\hat W}b(\tilde W,\hat W,\tilde P,\hat P)$

{\bf Question:} Is it possible to simplify the notation such that $\tilde w_i=w_i-\tilde w_i$? in such a way that once the blockchain operation is defined, the traditional transfers are defined.

{\bf Question:} Potential notation/normalization, $\hat w_i=\alpha w_i$ thus $\tilde w_i=(1-\alpha) w_i$ for $\alpha\in[0,1]$

Let us re-write \ref{eq:b} with the new notation:

\begin{eqnarray}
    b_i(W,\tilde P,\hat P)&=&(1-\alpha) w_i^T(\tilde \mu_i-\tilde Pe_i)-\gamma_i(1-\alpha) w_i^T\tilde \Sigma (1-\alpha) w_i\nonumber\\
            &+&\alpha w_i^T(\hat \mu_i-\hat Pe_i)-\gamma_i\alpha w_i^T\hat \Sigma \alpha w_i\label{eq:b}
\end{eqnarray}

The optimization problem is $\max_{\alpha, W}b(W,\tilde P, \hat P)$.


\subsection{Motivating example}

Firm A wants to transfer to firm K. In the traditional network the cost is $\tilde c$. There is a firm L such that A is able to transfer to L and L is able to transfer to K at a cost $\hat c<\tilde c$.
\section{Posible Problema de Maximización}

El agente optimiza dos decisiones clave:
\begin{itemize}
    \item $\alpha$: La proporción de la transacción destinada a blockchain.
    \item $w_i$: El tamaño total de la transacción.
\end{itemize}

Y tiene que resolver el siguiente problema de optimización:

\[
\max_{\alpha, w_i} U_i(W, \alpha, \tilde{P}, \hat{P}, N_{\text{blockchain}})
\]

\subsection{Función de Utilidad}

\[
U_i(W, \alpha, \tilde{P}, \hat{P}, N_{\text{blockchain}}) = (1 - \alpha) w_i^T (\tilde{\mu}_i - \tilde{P}e_i) - \gamma_i (1 - \alpha) w_i^T \tilde{\Sigma} (1 - \alpha) w_i
\]
\[
+ \alpha w_i^T (\hat{\mu}_i - \hat{P}e_i) - \gamma_i \alpha w_i^T \hat{\Sigma} \alpha w_i
\]
\[
- \gamma \cdot R_{\text{tradicional}}(w_i, H_{\text{blockchain}}) + \phi_i \cdot C_{\text{tradicional}} 
- \gamma \cdot \frac{H_{\text{blockchain}}(w_i, R_{\text{tradicional}})}{N_{\text{blockchain}}} + \eta_i \cdot N_{\text{blockchain}}
\]

\subsection{Explicación de términos y parámetros}

\begin{enumerate}
    \item \textbf{Distribución del contrato entre sistemas}:
    \begin{itemize}
        \item $(1 - \alpha) w_i$: Proporción del contrato total $w_i$ que el agente asigna al \textit{sistema tradicional}.
        \item $\alpha w_i$: Proporción del contrato total que el agente asigna a \textit{blockchain}.
        \item $\alpha$: Proporción de la transacción destinada a blockchain, con $\alpha \in [0, 1]$.
    \end{itemize}
    
    \item \textbf{Beneficios esperados}:
    \begin{itemize}
        \item \textit{Para el sistema tradicional}:
        \begin{itemize}
            \item $\tilde{\mu}_i$: Beneficios esperados por el agente en el sistema tradicional.
            \item $\tilde{P}$: Costos de transacción en el sistema tradicional (tarifas bancarias, costos de intermediación).
            \item $e_i$: Vector que representa al agente.
        \end{itemize}
        \item \textit{Para blockchain}:
        \begin{itemize}
            \item $\hat{\mu}_i$: Beneficios esperados en blockchain.
            \item $\hat{P}$: Costos de transacción en blockchain (comisiones de gas).
        \end{itemize}
    \end{itemize}

    \item \textbf{Riesgos financieros en cada sistema (ponderados por $\gamma_i$)}:
    \begin{itemize}
        \item \textit{Sistema tradicional}:
        \[
        - \gamma_i (1 - \alpha) w_i^T \tilde{\Sigma} (1 - \alpha) w_i
        \]
        Este término representa los \textit{riesgos financieros} (volatilidad) en el sistema tradicional, con $\tilde{\Sigma}$ como la matriz de covarianza de riesgos.
        
        \item \textit{Blockchain}:
        \[
        - \gamma_i \alpha w_i^T \hat{\Sigma} \alpha w_i
        \]
        Riesgos financieros en blockchain, ponderados por $\alpha$, donde $\hat{\Sigma}$ es la matriz de covarianza de los riesgos financieros en blockchain.
    \end{itemize}
    
    \item \textbf{Riesgos interdependientes entre sistemas}:
    \begin{itemize}
        \item $\gamma \cdot R_{\text{tradicional}}(w_i, H_{\text{blockchain}})$: Este término refleja el \textit{riesgo regulatorio} en el sistema tradicional, que depende del tamaño de la transacción ($w_i$) y del riesgo de hackeo en blockchain ($H_{\text{blockchain}}$).
        \item $\gamma \cdot \frac{H_{\text{blockchain}}(w_i, R_{\text{tradicional}})}{N_{\text{blockchain}}}$: Riesgo de hackeo en blockchain, que depende del tamaño de la red ($N_{\text{blockchain}}$) y el riesgo en el sistema tradicional.
    \end{itemize}
    
    \item \textbf{Confianza institucional en el sistema tradicional}:
    \[
    \phi_i \cdot C_{\text{tradicional}}
    \]
    Término que captura el nivel de \textit{confianza institucional} en las instituciones financieras tradicionales, que afecta positivamente la utilidad derivada de operar en ese sistema.
    
    \item \textbf{Efectos de red en blockchain}:
    \[
    \eta_i \cdot N_{\text{blockchain}}
    \]
    Término que mide el \textit{efecto de red} en blockchain. A medida que el número de participantes en la red ($N_{\text{blockchain}}$) crece, la seguridad y la liquidez aumentan, lo que incrementa la utilidad para los agentes.
\end{enumerate}
