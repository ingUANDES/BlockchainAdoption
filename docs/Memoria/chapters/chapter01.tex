% cap1.tex

\chapter{Introduction}\label{chapter:Intro} % la etiqueta para referencias

The adoption and implementation of blockchain technologies bring both significant opportunities and challenges across various industries and economic sectors. Blockchain can greatly improve security, transparency, and efficiency, but integrating it into existing systems is complex and needs careful study. Network theory provides a basis for understanding how different economic agents, like individuals, companies, and governments, form and maintain connections in blockchain adoption. This project aims to close the gap between theory and practice, using recent research on network stability, incentives, and information sharing to better understand how blockchain can be successfully implemented. 


\section{Goals}
\subsection{Main goal}


The primary objective of this thesis is to develop and analyze a theoretical model that explains how financial agents optimize the allocation of their transactions between traditional financial systems and blockchain-based systems. By incorporating key factors such as transaction costs, security risks, institutional trust, and network effects, this research seeks to provide a comprehensive understanding of the decision-making processes involved.

The model will examine the interplay between both systems, capturing the role of interdependent risks and the influence of network growth in blockchain. This study will determine the conditions under which financial agents reach an optimal equilibrium, balancing their use of blockchain and traditional systems, and will explore whether both systems can coexist or if one will eventually dominate. 

Ultimately, this research aims to offer insights into the future of financial systems, shedding light on how blockchain technologies may complement or disrupt traditional structures, and providing a robust framework for understanding the economic and strategic implications of this shift.


\subsection{Specific goals}

The specific goals of this thesis are to:

\begin{enumerate}
\item Develop a utility-based model that represents how financial agents allocate transactions between traditional financial systems and blockchain systems. This model will incorporate transaction costs, security risks, trust in institutions, and network effects to capture the decision-making process of the agents.

\item Analyze the role of transaction costs in both systems, determining how the costs associated with traditional intermediaries and blockchain networks affect the optimal allocation of financial resources by agents.

\item Investigate the influence of security and regulatory risks in both financial environments. The model will include risk factors such as the likelihood of hacking in blockchain systems and institutional risks in traditional systems, aiming to understand how agents manage these risks in their transaction decisions.

\item Evaluate the impact of network effects in blockchain systems, particularly how the growth of the blockchain network can enhance security, reduce costs, and increase liquidity, thereby making it more attractive for financial agents. 

\item Explore the interdependence of risks between blockchain and traditional financial systems, focusing on how increased risks in one system may affect the stability, costs, or adoption of the other system.

\item Derive conditions for equilibrium in which financial agents reach a stable allocation of transactions between the two systems, or alternatively, identify the circumstances in which one system may dominate over the other.

\item Provide a comprehensive framework that explains the economic and strategic implications of blockchain adoption, offering insights into whether blockchain will complement or disrupt traditional financial systems, and under what conditions each system can thrive.

\item Understand how information spreads and social learning occurs regarding blockchain
technologies.

\item Assess how cryptocurrency price uncertainty affects the provision of blockchain net-
works as public goods.
\end{enumerate}


\section{Alcances}

This project will focus on analyzing the adoption and implementation of blockchain technologies within various industries and economic sectors. The study will leverage network theory to understand how different stakeholders—such as individuals, companies, and governments—form and maintain connections within blockchain networks. It will specifically examine the incentives and barriers influencing stakeholders’ decisions to adopt blockchain technology, as well as the processes of information diffusion and social learning about these technologies. Additionally, the research will assess how fluctuations in cryptocurrency prices impact the sustainability and provision of blockchain as a public good within decentralized networks. Practical applications and challenges of blockchain in transforming existing network structures will also be explored.
\\
\\
However, this research will not cover all industries or sectors, focusing instead on those where blockchain adoption is most relevant and impactful. It will not consider all potential factors affecting blockchain adoption, such as political or cultural influences, but will concentrate primarily on economic and network-related dynamics. The study will not provide an exhaustive historical analysis of blockchain technology, but will focus on recent trends and developments. Furthermore, while the findings will offer valuable insights, they may not be broadly generalizable without further validation, and will be most applicable to the specific industries, sectors, and regions examined in this study.



\section{Methodology}


\begin{enumerate}
\item Literature review
\item Description of the theory related
\item Model proposal
\item Characterization of the model
\end{enumerate}

\newpage
\section{Estructura del documento}

describir la estructura general del documento global



