% cap1.tex

\chapter{Introduction}\label{chapter:Intro} % la etiqueta para referencias

The adoption and implementation of blockchain technologies bring both significant opportunities and challenges across various industries and economic sectors. Blockchain can greatly improve security, transparency, and efficiency, but integrating it into existing systems is complex and needs careful study. Network theory provides a basis for understanding how different economic agents, like individuals, companies, and governments, form and maintain connections in blockchain adoption. This project aims to close the gap between theory and practice, using recent research on network stability, incentives, and information sharing to better understand how blockchain can be successfully implemented. 


\section{Goals}
\subsection{Main goal}

The primary objective of this project is to explore and analyze the adoption and implementation of blockchain technologies within various industries and economic sectors, using network theory and competitive markets as a conceptual framework. By examining the interactions, connections, and dynamics among economic agents, this research aims to provide a comprehensive understanding of how blockchain technologies can be effectively integrated into existing systems, addressing both the opportunities and challenges involved.

\subsection{Specific goals}

\begin{enumerate}
\item Identify the structural properties of blockchain adoption networks and determine the conditions that lead to their stability and efficiency.

\item Evaluate the incentives and barriers that different stakeholders face when adopting blockchain technology.

\item Understand how information spreads and social learning occurs regarding blockchain technologies.

\item Assess how cryptocurrency price uncertainty affects the provision of blockchain networks as public goods.

\item Explore the practical applications and challenges of using blockchain technologies to transform network structures.
\end{enumerate}


\section{Alcances}

This project will focus on analyzing the adoption and implementation of blockchain technologies within various industries and economic sectors. The study will leverage network theory to understand how different stakeholders—such as individuals, companies, and governments—form and maintain connections within blockchain networks. It will specifically examine the incentives and barriers influencing stakeholders’ decisions to adopt blockchain technology, as well as the processes of information diffusion and social learning about these technologies. Additionally, the research will assess how fluctuations in cryptocurrency prices impact the sustainability and provision of blockchain as a public good within decentralized networks. Practical applications and challenges of blockchain in transforming existing network structures will also be explored.
\\
\\
However, this research will not cover all industries or sectors, focusing instead on those where blockchain adoption is most relevant and impactful. It will not consider all potential factors affecting blockchain adoption, such as political or cultural influences, but will concentrate primarily on economic and network-related dynamics. The study will not provide an exhaustive historical analysis of blockchain technology, but will focus on recent trends and developments. Furthermore, while the findings will offer valuable insights, they may not be broadly generalizable without further validation, and will be most applicable to the specific industries, sectors, and regions examined in this study.



\section{Methodology}


\begin{enumerate}
\item Literature review
\item Description of the theory related
\item Model proposal
\item Characterization of the model
\end{enumerate}

\newpage
\section{Estructura del documento}

describir la estructura general del documento global



