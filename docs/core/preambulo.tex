% Definiciones basicas de LaTeX para la memoria
%
% No modificar las lineas a continuacion a menos
% de estar muy seguro de lo que se esta haciendo!!
%
% Jaime Cisternas, julio 2004, agosto 2010

% <jcisternas@uandes.cl> 

%%%%%%%%%%%%%%%%%%%%%%%%%%%%%%%%%%%%%%%%%%%%%%%%%%%%%%%%%%%%%%%%%%%%%
% Libreria computin
%\usepackage{algorithm}
%\usepackage{algorithmic}
\usepackage[left=3cm,top=2.5cm,right=3cm,bottom=2.5cm]{geometry}

% Encoding

%\usepackage[utf8]{inputenc} % unnecessary when compiling through xelatex
\usepackage[T1]{fontenc}

% Carga librerias utiles con simbolos y estructuras
\usepackage{amsfonts}
\usepackage{amssymb}
\usepackage{amsmath}   
\usepackage{amsthm}
\usepackage{latexsym}
\usepackage{cancel}
\usepackage{physics}
\usepackage{bbm}
\usepackage{listings}
\usepackage{pgfplots}
\usepackage{eurosym}

% Logo Julia

\usepackage{core/julialogo}

% Cool code highlight
% https://es.overleaf.com/learn/latex/Code_Highlighting_with_minted
% No need of outputdir out of overleaf
\usepackage[outputdir=docs/DocumentoMemoria]{minted}

% Librerías adicionales para diagramas tikz

\usepackage{tikz}
\usepackage{mathdots}
\usepackage{yhmath}
\usepackage{color}
\usepackage{xcolor}
\usepackage{siunitx}
\usepackage{multirow}
\usepackage{gensymb}
\usepackage{tabularx}
\usepackage{extarrows}
\usepackage{booktabs}
\usetikzlibrary{fadings}
\usetikzlibrary{patterns}
\usetikzlibrary{shadows.blur}
\usetikzlibrary{shapes}


% Libreria para ajustar espaciado entre lineas
\usepackage{setspace}

% Libreria con estilo APA para bibliografia
\usepackage{core/newapa}

% Si usted está utilizando un teclado en español, puede ser útil
% usar los paquetes a continuacion. De otra manera tendrá que generar las
% tildes y la letra ñ de forma especial.
% Estos caracteres no pueden ser utilizados en etiquetas ni tampoco en modo matemático
\usepackage[spanish, activeacute]{babel}
\unaccentedoperators


% Declaraciones especificas de pdfLaTeX
\usepackage[%pdftex, % commented to compile with Xelatex
        colorlinks=false,         % true or false (for final version)
        urlcolor=rltblue,         % \href{...}{...} external (URL)
	    anchorcolor=rltbrightblue,
        filecolor=weben,          % \href*{...} local file
        linkcolor=webred,         % \ref{...} and \pageref{...}
        menucolor=webdarkblue,
        citecolor=webgreen,
        pdftitle={},              % INSERT YOUR TITLE HERE
        pdfauthor={},             % INSERT YOUR NAME HERE
        pdfsubject={},
        pdfkeywords={},
        %pdfpagemode=None,
        bookmarksopen=true,
	plainpages=false]{hyperref}
%\usepackage[pdftex]{graphicx}

\usepackage{graphicx}

\usepackage{fontspec}
 
\setmainfont{Times New Roman}



%\pdfcompresslevel=9 %commented to use Xelatex
%\pdfadjustspacing=1 %commented to use xelatex
% Definicion de colores
\usepackage{color}
\definecolor{rltbrightred}{rgb}{1,0,0}
\definecolor{rltred}{rgb}{0.75,0,0}
\definecolor{rltdarkred}{rgb}{0.5,0,0}
\definecolor{rltbrightgreen}{rgb}{0,0.75,0}
\definecolor{rltgreen}{rgb}{0,0.5,0}
\definecolor{rltdarkgreen}{rgb}{0,0,0.25}
\definecolor{rltbrightblue}{rgb}{0,0,1}
\definecolor{rltblue}{rgb}{0,0,0.75}
\definecolor{rltdarkblue}{rgb}{0,0,0.5}
\definecolor{webred}{rgb}{0.5,.25,0}
\definecolor{webblue}{rgb}{0,0,0.75}
\definecolor{webgreen}{rgb}{0,0.5,0}
\definecolor{webdarkblue}{rgb}{0,0,0.5}
\definecolor{webbrightgreen}{rgb}{0,0.75,0}
% fin de declaraciones especificas pdfLaTeX

% tamanio de pagina
%\setlength{\oddsidemargin}{.5in}
%\setlength{\evensidemargin}{.0in} % en caso de usar opcion 'twoside'
%\setlength{\textwidth}{6in}
%\setlength{\topmargin}{-.5in}
%\setlength{\textheight}{9in}


% Las siguientes definiciones pueden ser usadas en
% memorias mas matematicas
\newtheorem{theorem}{Teorema}[section]
\newtheorem{lemma}[theorem]{Lema}
\newtheorem{corollary}[theorem]{Corolario}
\newtheorem{proposition}[theorem]{Proposición}
\newtheorem{definition}[theorem]{Definición}
\newtheorem{claim}{Afirmación}
\newtheorem{conjecture}[theorem]{Conjetura}
\newtheorem{observation}[theorem]{Observación}
\newtheorem{problem}[theorem]{Problema}

% Definicion de un ambiente para algoritmos en pseudo-codigo.
% Esta definicion puede ser mejorada.
\newtheorem{algorithm}{Algoritmo}[section]
\newcommand{\tab}{\hspace*{0.5 cm}}
% Palablas claves que deben aparecer en negrita
\newcommand{\bfWHILE}{\textbf{while~}}
\newcommand{\bfDO}{\textbf{do~}}
\newcommand{\bfEND}{\textbf{end~}}
\newcommand{\bfIF}{\textbf{if~}}
\newcommand{\bfTHEN}{\textbf{then~}}
\newcommand{\bfELSE}{\textbf{else~}}
\newcommand{\bfFOR}{\textbf{for~}}

% Nombres fijos de Latex pueden ser cambiados al Espaniol
\renewcommand{\contentsname}{Índice General}
\renewcommand{\chaptername}{Capítulo}
\renewcommand{\appendixname}{Anexo}
\renewcommand{\bibname}{Bibliografía}
\renewcommand{\figurename}{Ilustración}
\renewcommand{\tablename}{Tabla}
\renewcommand{\indexname}{Índice}
\renewcommand{\partname}{Parte}
\renewcommand{\listfigurename}{Lista de Ilustraciones}
\renewcommand{\listtablename}{Lista de Tablas}

% Para las enumeraciones usamos primero a,b,c,... y despues i, ii, iii,...
\renewcommand{\labelenumi}{\alph{enumi})}
\renewcommand{\labelenumii}{\roman{enumii})}
\renewcommand{\labelenumiii}{-}
\renewcommand{\labelenumiv}{-}

% Para ser consistentes en el uso de abreviaciones!
%\newcommand{\Chp}{Cap\'{\i}tulo}
%\newcommand{\Chps}{Cap\'{\i}tulos}
%\newcommand{\Sec}{Sec.} % \S
%\newcommand{\Secs}{Secs.} % \S\S
%\newcommand{\SSec}{Subsec.}
%\newcommand{\Fig}{Fig.}
%\newcommand{\Figs}{Figs.}
%\newcommand{\Eqn}{Ecn.}
%\newcommand{\Eqns}{Ecns.}

% Muestra en pantalla los nombres de los archivos usados
\listfiles

% Previene una cierta senial de ``warning: contentsline with no destination''
\newcounter{dummy}



\setlength{\parindent}{0cm}
\usepackage{float}
\usepackage{subfigure}
\usepackage{array}
\newcolumntype{E}{>{$}c<{$}}
\usepackage{longtable}
\setcounter{MaxMatrixCols}{40}
\usepackage{bm}
